\documentclass[openany, 11pt]{book}
\usepackage[inkscapeformat=png]{svg}
\makeindex

\usepackage{amsmath}
\usepackage{amssymb}
\usepackage{booktabs}
\usepackage{csvsimple-l3}
\usepackage{bussproofs}
\usepackage{tabularray}
\usepackage{dirtytalk}
\usepackage[dvipsnames]{xcolor}
\usepackage{enumitem}
\usepackage{epigraph}
\usepackage{forest}
\usepackage{formal-grammar}
\usepackage{graphicx}
\usepackage[hyperindex]{hyperref}
\usepackage{hyperref}
\usepackage{kantlipsum}
\usepackage{makeidx}
\usepackage[margin=0.8in]{geometry}% Sets 1in margins.
\usepackage{mathrsfs}
\usepackage{minted}
\usepackage{multicol}
\usepackage{standalone}
\usepackage[style=authortitle]{biblatex}
\usepackage[T1]{fontenc}
\usepackage[tableaux]{prooftrees}
\usepackage{tcolorbox}
\usepackage{tikz}
\usepackage{titlesec}
\usepackage{xcolor}

\usetikzlibrary{arrows}
\usetikzlibrary{arrows.meta}
\usetikzlibrary{automata}
\usetikzlibrary{calc}
\usetikzlibrary{fit}
\usetikzlibrary{petri}
\usetikzlibrary{positioning}

\tcbuselibrary{breakable}
\tcbuselibrary{listings}
\tcbuselibrary{minted}
\tcbuselibrary{skins}
\tcbuselibrary{theorems}

\newcounter{filePrg}

\addbibresource{biblio.bib}
\setlength{\parindent}{0pt}

\renewcommand{\emph}[1]{\textit{#1}}
\setlength{\parindent}{0pt}

\setlength{\parindent}{10pt}
\newcommand{\set}[1]{\{#1\}}

\definecolor{CaribbeanBlue}{RGB}{0, 206, 209} % Define Caribbean Blue
\NewTcbTheorem[list inside=definition]{definition}
{Definition}{
	breakable,
	colback=CaribbeanBlue!05,
	colframe=CaribbeanBlue!35!black,
	fonttitle=\bfseries}{th}

\NewTcbTheorem[list inside=intuition]{intuition}{Intuition}{
	breakable,
	colback=blue!5,
	colframe=blue!35!black,
	fonttitle=\bfseries}{th}

\NewTcbTheorem{example}{Example}{
	breakable,
	colback=white,
	colframe=green!35!black,
	fonttitle=\bfseries}{th}

\NewTcbTheorem{verify}{Verify}{
	breakable,
	float,
	colback=red!5,
	colframe=red!35!black,
	fonttitle=\bfseries}{th}

\NewTcbTheorem[list inside=theorem]{theorem}{Theorem}{
	breakable,
	colback=gray!10,
	colframe=gray!35!black,
	fonttitle=\bfseries}{th}

\NewTcbTheorem[
	list inside=exercise,
	number within=section
]
{exercise}{Exercise}{
	breakable,
	colback=white,
	colframe=black,
	fonttitle=\bfseries}{th}

\newcommand{\hask}[1]{\mintinline{haskell}{#1}}

\renewcommand{\thesection}{\arabic{section}}
\tcbset{enhanced jigsaw}

\newtcbinputlisting{\codeFromFile}[2]{
	listing file={#1},
	listing engine=minted,
	minted style=colorful,
	minted language=haskell,
	minted options={breaklines,linenos,numbersep=3mm},
	colback=blue!5!white,colframe=blue!75!black,listing only,
	left=5mm,enhanced,
	title={#2},
	overlay={\begin{tcbclipinterior}\fill[red!20!blue!20!white] (frame.south west)
				rectangle ([xshift=5mm]frame.north west);\end{tcbclipinterior}}
}

\newtcblisting{haskell}[1]
{
	listing engine=minted,
	minted style=colorful,
	minted language=haskell,
	minted options={breaklines,linenos,numbersep=3mm},
	colback=blue!5!white,colframe=blue!75!black,listing only,
	left=5mm,enhanced,
	title={#1},
	overlay={\begin{tcbclipinterior}\fill[red!20!blue!20!white] (frame.south west)
				rectangle ([xshift=5mm]frame.north west);\end{tcbclipinterior}}
}

\title{Haskell Numbers}
\author{Idris}
\date{December 2024}

\begin{document}
% \maketitle{}
\tableofcontents
% \tcblistof[\section]{definition}{List of Definitions}
% \tcblistof[\section]{intuition}{List of Intuitions}

% \listoffigures
% \listoftables

\chapter{Type Classes}
\section{Definitions}
\begin{definition}{Background}{}
	\begin{enumerate}[label = {(\arabic*)}]
		\item class hierarchy is mostly historical
		\item if redone now, would be done in more systematic way
		\item but cant be done, would break a ton of code
	\end{enumerate}
\end{definition}

\begin{definition}{Type Classes}{}
	\begin{multicols}{2}
		\begin{enumerate}[label = {(\arabic*)}]
			\item Bounded
			\item Enum
			\item Eq
			\item Floating
			\item Fractional
			\item Integral
			\item Num
			\item Ord
			\item Real
			\item RealFloat
			\item RealFrac
		\end{enumerate}
	\end{multicols}
\end{definition}

\begin{definition}{Types}{}
	\begin{multicols}{2}
		\begin{enumerate}[label = {(\arabic*)}]
			\item Bool
			\item Char
			\item Complex
			\item Double
			\item Float
			\item Int
			\item Integer
			\item Ordering
			\item Word
		\end{enumerate}
	\end{multicols}
\end{definition}



\begin{intuition}{Real}{}
	\begin{haskell}{}
type Real :: * -> Constraint
class (Num a, Ord a) => Real a where
toRational :: a -> Rational
    {-# MINIMAL toRational #-}
instance Real Double -- Defined in 'GHC.Float’
instance Real Float -- Defined in 'GHC.Float’
instance Real Int -- Defined in 'GHC.Real’
instance Real Integer -- Defined in 'GHC.Real’
instance Real Word -- Defined in 'GHC.Real’
	\end{haskell}
\end{intuition}

\begin{intuition}{Bounded}{}
	Max and Min, but not necessarily any order.
	\begin{haskell}{}
class Bounded a where
    minBound :: a
    maxBound :: a
	\end{haskell}
\end{intuition}

\begin{intuition}{Enum}{}
	\hask{pred} and \hask{succ}, but not necessarily a max or min.
\end{intuition}

\begin{intuition}{Integral}{}
	It's a ring.
	\begin{haskell}{}
class (Real a, Enum a) => Integral a where
    quot :: a -> a -> a
    rem :: a -> a -> a
    div :: a -> a -> a
    mod :: a -> a -> a
    quotRem :: a -> a -> (a, a)
    divMod :: a -> a -> (a, a)
    toInteger :: a -> Integer
    \end{haskell}
\end{intuition}

\begin{intuition}{RealFrac}{}
	Gives you rounding. It's like a way to introduce Integral.
	Basically you go from \hask{RealFrac} to \hask{Integral}.
	\begin{haskell}{}
type RealFrac :: * -> Constraint

class (Real a, Fractional a) => RealFrac a where
    properFraction :: Integral b => a -> (b, a)
    truncate :: Integral b => a -> b
    round :: Integral b => a -> b
    ceiling :: Integral b => a -> b
    floor :: Integral b => a -> b
    {-# MINIMAL properFraction #-}

instance RealFrac Double -- Defined in 'GHC.Float’
instance RealFrac Float -- Defined in 'GHC.Float’
\end{haskell}
\end{intuition}

\begin{intuition}{Fractional}{}
	So since it starts with Num, then gets inverse, its a \ldots group?!
	\begin{haskell}{}
class Num a => Fractional a where
    (/) :: a -> a -> a
    recip :: a -> a
    fromRational :: Rational -> a
    {-# MINIMAL fromRational, (recip | (/)) #-}

instance Fractional Double -- Defined in 'GHC.Float’
instance Fractional Float -- Defined in 'GHC.Float’
instance RealFloat a => Fractional (Complex a)
\end{haskell}
\end{intuition}

\begin{intuition}{Floating}{}
	Trigonometric like functions. Notice that these are all a to a, so these
	types have to respect closure.
	\begin{align*}
		\cos, \sin, \tan, \pi, \sqrt{}
	\end{align*}
	\begin{haskell}{}
class Fractional a => Floating a where
    pi :: a
    exp :: a -> a
    log :: a -> a
    sqrt :: a -> a
    (**) :: a -> a -> a
    logBase :: a -> a -> a
    ...
    \end{haskell}
\end{intuition}

\begin{intuition}{RealFloat}{}
	Allows you to separate float in radix and the other thing?

	\begin{haskell}{}
class (RealFrac a, Floating a) => RealFloat a where
  floatRadix :: a -> Integer
  floatDigits :: a -> Int
  floatRange :: a -> (Int, Int)
  decodeFloat :: a -> (Integer, Int)
  encodeFloat :: Integer -> Int -> a
  exponent :: a -> Int
  significand :: a -> a
  scaleFloat :: Int -> a -> a
  isNaN :: a -> Bool
  isInfinite :: a -> Bool
  isDenormalized :: a -> Bool
  isNegativeZero :: a -> Bool
  isIEEE :: a -> Bool
  atan2 :: a -> a -> a
\end{haskell}
\end{intuition}


\begin{intuition}{Complex}{}
	Has kind \texttt{Type $\rightarrow$ Type}.
	\begin{haskell}{}
type Complex :: * -> *
data Complex a = !a :+ !a
-- Defined in 'Data.Complex’
instance RealFloat a => Floating (Complex a)
-- Defined in 'Data.Complex’
instance Foldable Complex -- Defined in 'Data.Complex’
instance Traversable Complex -- Defined in 'Data.Complex’
instance Read a => Read (Complex a) -- Defined in 'Data.Complex’
instance RealFloat a => Fractional (Complex a)
-- Defined in 'Data.Complex’
instance RealFloat a => Num (Complex a)
-- Defined in 'Data.Complex’
instance Show a => Show (Complex a) -- Defined in 'Data.Complex’
instance Applicative Complex -- Defined in 'Data.Complex’
instance Functor Complex -- Defined in 'Data.Complex’
instance Monad Complex -- Defined in 'Data.Complex’
instance Eq a => Eq (Complex a) -- Defined in 'Data.Complex’
\end{haskell}
\end{intuition}

\section{Kinds}

\begin{center}
	\rowcolors{2}{gray!10}{white}
	\begin{tabular}{lccccccccc}
		\toprule
		Class      & Bool         & Char         & Complex      & Double       & Float        & Int          & Integer      & Ordering     & Word         \\
		\midrule
		Bounded    & \checkmark{} & \checkmark{} &              &              &              & \checkmark{} &              & \checkmark{} & \checkmark{} \\
		Enum       & \checkmark{} & \checkmark{} &              & \checkmark{} & \checkmark{} & \checkmark{} & \checkmark{} & \checkmark{} & \checkmark{} \\
		Eq         & \checkmark{} & \checkmark{} & \checkmark{} & \checkmark{} & \checkmark{} & \checkmark{} & \checkmark{} & \checkmark{} & \checkmark{} \\
		Floating   &              &              & \checkmark{} & \checkmark{} & \checkmark{} &              &              &              &              \\
		Fractional &              &              & \checkmark{} & \checkmark{} & \checkmark{} &              &              &              &              \\
		Integral   &              &              &              &              &              & \checkmark{} & \checkmark{} &              & \checkmark{} \\
		Num        &              &              & \checkmark{} & \checkmark{} & \checkmark{} & \checkmark{} & \checkmark{} &              & \checkmark{} \\
		Ord        & \checkmark{} & \checkmark{} &              & \checkmark{} & \checkmark{} & \checkmark{} & \checkmark{} & \checkmark{} & \checkmark{} \\
		Real       &              &              &              & \checkmark{} & \checkmark{} & \checkmark{} & \checkmark{} &              & \checkmark{} \\
		RealFloat  &              &              & \checkmark{} & \checkmark{} & \checkmark{} &              &              &              &              \\
		RealFrac   &              &              &              & \checkmark{} & \checkmark{} &              &              &              &              \\
		\bottomrule
	\end{tabular}
\end{center}

\begin{center}
	\begin{tabular}{lll}
		\toprule
		Function            & Type Signature                          & Notes                             \\
		\midrule
		\hask{fromIntegral} & \hask{(Integral a, Num b) => a -> b}    & General integral $\to$ numeric    \\
		\hask{realToFrac}   & \hask{(Real a, Fractional b) => a -> b} & General real $\to$ fractional     \\
		\hask{toInteger}    & \hask{Integral a => a -> Integer}       & Narrow to \hask{Integer}          \\
		\hask{toRational}   & \hask{Real a => a -> Rational}          & Narrow to \hask{Rational}         \\
		\hask{fromEnum}     & \hask{Enum a => a -> Int}               & Enum $\to$ \hask{Int}             \\
		\hask{toEnum}       & \hask{Enum a => Int -> a}               & \hask{Int} $\to$ Enum             \\
		\hask{fromInteger}  & \hask{Num a => Integer -> a}            & Class method of \hask{Num}        \\
		\hask{fromRational} & \hask{Fractional a => Rational -> a}    & Class method of \hask{Fractional} \\
		\bottomrule
	\end{tabular}
\end{center}

\begin{figure}[H]
	\centering
	\begin{tikzcd}[row sep=3em, column sep=6em]
		\text{Integral} \arrow[dr, "toInteger"'] \arrow[r, "fromIntegral"]
		& \text{Num} \\
		& \text{Integer} \arrow[u, "fromInteger"']
	\end{tikzcd}
	\hspace{4em}
	\begin{tikzcd}[row sep=3em, column sep=6em]
		\text{Real} \arrow[dr, "toRational"'] \arrow[r, "realToFrac"]
		& \text{Fractional} \\
		& \text{Rational} \arrow[u, "fromRational"']
	\end{tikzcd}
	\caption{Core Haskell numeric coercions: Integral \(\leftrightarrow\) Num, and Real \(\leftrightarrow\) Fractional.}
\end{figure}

\begin{figure}[H]
	\begin{center}
		\includesvg{classes.svg}
	\end{center}
	\caption{Number Class Hierarchy}
\end{figure}

% \printbibliography{}
% \printindex{}
\end{document}
